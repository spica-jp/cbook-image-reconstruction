\documentclass{ltjsarticle}
\usepackage[strict,showonlyrefs,notwithinsection]{skrep}
\tcbset{%
  important/.style={%
    enhanced,frame hidden,colback=gray!20,arc=1.5mm,%
  }
}

\begin{document}
\title{『C言語による画像再構成の基礎』式(4-53)に関して}
\author{@spica-jp}
\date{\today(\jpndow)}
\maketitle
教科書(『C言語による画像再構成の基礎』)p. 144の式(4-53)の2行上で提示された式,すなわち
\begin{equation}
F(2k+1)=\sum_{j=0}^{\pffrac{N}{2}-1}f'(j)\exp(-\frac{2\pi ij(2k+1)}{N})+\sum_{j=0}^{\pffrac{N}{2}-1}f'\qty(j+\frac{N}{2})\exp(-\frac{2\pi i(j+\pffrac{N}{2})(2k+1)}{N})
\end{equation}
から出発する。\par
さて,
\begin{align}
\exp(-\frac{2\pi ij(2k+1)}{N})
&=\exp(-\frac{4\pi ijk+2\pi ij}{N})\\
&=\exp(-\frac{4\pi ijk}{N}-\frac{2\pi ij}{N})\\
&=\exp(-\frac{4\pi ijk}{N})\exp\qty(-\frac{2\pi ij}{N})\\
&=\exp(-\frac{2\pi ijk}{\ffrac{N}{2}})\exp\qty(-\frac{2\pi ij}{N})
\end{align}
であり,また,
\begin{align}
\exp(-\frac{2\pi i(j+\pffrac{N}{2})(2k+1)}{N})
&=\exp(-\frac{(2\pi ij+\pi iN)(2k+1)}{N})\\
%&=\exp(-\frac{4\pi ijk+2\pi ij+\pi iN(2k+1)}{N})\\
&=\exp(-\frac{4\pi ijk+2\pi ij}{N}-\frac{\pi iN(2k+1)}{N})\\
&=\exp(-\frac{4\pi ijk}{N}-\frac{2\pi ij}{N}-\pi i(2k+1))\\
&=\exp(-\frac{4\pi ijk}{N})\exp(-\frac{2\pi ij}{N})\exp(-\pi i(2k+1))\\
&=\exp(-\frac{2\pi ijk}{\ffrac{N}{2}})\exp(-\frac{2\pi ij}{N})\exp(-\pi i(2k+1))
\end{align}
である。\par
よって,
\begin{align}
F(2k+1)
&=\sum_{j=0}^{\pffrac{N}{2}-1}f'(j)\exp(-\frac{2\pi ijk}{\ffrac{N}{2}})\exp\qty(-\frac{2\pi ij}{N})\\
&\phantom{{}={}}+\sum_{j=0}^{\pffrac{N}{2}-1}f'\qty(j+\frac{N}{2})\exp(-\frac{2\pi ijk}{\ffrac{N}{2}})\exp(-\frac{2\pi ij}{N})\exp(-\pi i(2k+1))\\
&=\sum_{j=0}^{\pffrac{N}{2}-1}f'(j)\exp(-\frac{2\pi ijk}{\ffrac{N}{2}})\exp(-\frac{2\pi ij}{N})\\
&\phantom{{}={}}+\exp(-\pi i(2k+1))\sum_{j=0}^{\pffrac{N}{2}-1}f'\qty(j+\frac{N}{2})\exp(-\frac{2\pi ijk}{\ffrac{N}{2}})\exp(-\frac{2\pi ij}{N})\\
&=\sum_{j=0}^{\pffrac{N}{2}-1}\qty(f'(j)+\exp(-\pi i(2k+1))f'\qty(j+\frac{N}{2}))\exp(-\frac{2\pi ijk}{\ffrac{N}{2}})\exp(-\frac{2\pi ij}{N})
\end{align}
となる。ここで,オイラーの公式\footnote{%
$\theta(\in\mathbb{R})$に対し,$e^{i\theta}=\cos\theta+i\sin\theta$が成り立つ,という公式であり,教科書ではp. 140にて言及されている。
},及び$k\in\mathbb{Z}$より,
\begin{equation}
\exp(-\pi i(2k+1))
=\cos(-\pi(2k+1))+i\sin(-\pi(2k+1))
=-1+0
=-1
\end{equation}
であるから,
\begin{equation}
\tcboxmath[important]{%
F(2k+1)=\sum_{j=0}^{\pffrac{N}{2}-1}\qty(f'(j)-f'\qty(j+\frac{N}{2}))\exp(-\frac{2\pi ijk}{\ffrac{N}{2}})\exp(-\frac{2\pi ij}{N})
}
\label{eq:correct}
\end{equation}
を得る。これを教科書風に書き直せば,
\begin{equation}
F(2k+1)=\sum_{j=0}^{\ffrac{N}{2}-1}\qty[f'(j)-f'(j+\ffrac{N}{2})]e^{\textstyle-\frac{2\pi ijk}{\ffrac{N}{2}}}\cdot e^{\textstyle-\frac{2\pi ij}{N}}
\end{equation}
であり,教科書の式(4-53),すなわち
\begin{equation}
F(2k+1)=\sum_{j=0}^{\ffrac{N}{2}-1}\qty[f'(j)-e^{\textstyle-\frac{2\pi ij}{N}}\cdot f'(j+\ffrac{N}{2})]e^{\textstyle-\frac{2\pi ijk}{\ffrac{N}{2}}}
\end{equation}
とはならない。\par
なお,式\eqref{eq:correct}をさらに整理すると,
\begin{equation}
\tcboxmath[important]{%
F(2k+1)=\sum_{j=0}^{\pffrac{N}{2}-1}\qty(f'(j)-f'\qty(j+\frac{N}{2}))\exp(-\frac{\pi ij(2k+1)}{\ffrac{N}{2}})
}
\end{equation}
となる。教科書の式(4-52),すなわち
\begin{equation}
F(2k)=\sum_{j=0}^{\pffrac{N}{2}-1}\qty(f'(j)+f'\qty(j+\frac{N}{2}))\exp(-\frac{\pi ij(2k)}{\ffrac{N}{2}})
\end{equation}
とも比較されたい。
\end{document}